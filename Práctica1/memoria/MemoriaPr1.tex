\documentclass[12pt]{article}

\usepackage{lmodern}
\usepackage[T1]{fontenc}
\usepackage[spanish,activeacute]{babel}
\usepackage[utf8]{inputenc}
\usepackage{mathtools}
\usepackage{enumerate}
\usepackage{amsthm}
\usepackage{amssymb}
\usepackage[hidelinks]{hyperref}
\usepackage{anysize}
\usepackage{listings}
\usepackage{float}

\marginsize{2cm}{2cm}{2cm}{2cm}

\lstset{ %
escapeinside=||,
language=python,
basicstyle=\small}

\title{Metaheur\'isticas:\\
 Pr\'actica 1.b: B\'usquedas por Trayectorias para el Problema de la Selecci\'on de Caracter\'isticas}
\author{Anabel G\'omez R\'ios.\\
 DNI: 75929914Z.\\
 E-mail: anabelgrios@correo.ugr.es}


\begin{document}
\maketitle

\begin{center}
Curso 2015-2016\\

Problema de Selección de Características.\\ 

Grupo de prácticas: Viernes 17:30-19:30\\

Quinto curso del Doble Grado en Ingeniería Informática y Matemáticas.\\
\textit{ }\\
\end{center}

Algoritmos considerados:
\begin{enumerate}
\item Greedy Sequential Forward Selection.
\item Búsqueda Local.
\item Enfriamiento Simulado.
\item Búsqueda Tabú Básica.
\item Búsqueda Tabú Extendida.
\end{enumerate}

\newpage

\tableofcontents

\newpage

\section{Descripción del problema}
Queremos obtener un sistema que permita clasificar un conjunto de objetos en unas determinadas clases que conocemos previamente. Para ello disponemos de una muestra de dichos objetos ya clasificados y una serie de características para cada objeto.\\

El problema es, por tanto, construir un clasificador que se comporte lo suficientemente bien fuera de la muestra de la que disponemos, es decir, que clasifique bien nuevos datos. Para hacer esto, lo que hacemos es particionar la muestra en dos subconjuntos, uno que utilizaremos de entrenamiento para que el clasificador aprenda y otro que utilizaremos para test, es decir, para ver cómo de bien se comporta el clasificador que hemos obtenido con el primer subconjunto fuera de los datos de entrenamiento. Además haremos distintas particiones, en concreto 5 para mayor seguridad y construiremos un clasificador para cada una de ellas. Podemos comprobar cómo de bien se comporta cada clasificador porque sabemos en todo momento las clases de los objetos que tenemos en la muestra y podemos comparar las verdaderas clases con las que el clasificador obtiene.\\
Buscamos pues en todo momento optimizar la tasa de acierto del clasificador.\\
Vamos a utilizar además validación cruzada: es decir, para cada partición en dos subconjuntos primero uno será el de entrenamiento y el otro el de test y después les daremos la vuelta y volveremos a construir un clasificador. La calidad por tanto de cada método será la media de los porcentajes de clasificación (la tasa de acierto) para estas 10 particiones.\\

Nos queda describir cómo aprende el clasificador con los datos de entrenamiento. Ya que podemos llegar a tener muchas características de las cuales podríamos tener algunas poco o nada significantes, lo que hacemos es elegir un subconjunto de características que describan bien los datos de entrenamiento, de forma que en los datos de test sólo tenemos en cuenta este subconjunto de características a la hora de deducir cuál es la clase de cada nuevo dato. Para esta "deducción" vamos a utilizar la técnica de los 3 vecinos más cercanos: buscamos para cada dato los tres vecinos más cercanos teniendo en cuenta las características seleccionadas hasta el momento y nos quedamos con la clase que más veces aparezca. El cómo exploramos el espacio de búsqueda hasta encontrar la mejor solución (el subconjunto de características óptimo) es en lo que se diferencian los distintos algoritmos que vamos a ver en esta práctica.

\newpage

\section{Descripción de la aplicación de los algoritmos empleados al problema}
El primer paso común a todos los algoritmos es normalizar los datos de los que disponemos por columnas (es decir, por características) de forma que todas se queden entre 0 y 1 y no haya así preferencias de unas sobre otras.\\

A continuación se genera una solución inicial, que estará en todos los casos (menos en el algoritmo greedy, que se empieza desde cero) generada aleatoriamente y se irán generando vecinos de esta solución (o se darán saltos, como ya veremos) y nos quedaremos con la mejor solución obtenida.\\

Todos los algoritmos tienen una condición de parada común, que es llegar a un número máximo de evaluaciones o soluciones generadas: 15000.

\subsection{Representación de soluciones}
La representación elegida para las soluciones ha sido binaria: un vector de $N$ posiciones, donde $N$ es el número de características, en el que aparece \textit{True} o \textit{False} en la posición $i$ según si la característica $i$-ésima ha sido seleccionada o no, respectivamente.

\subsection{3NN}
Como hemos comentado, la técnica para clasificar que vamos a utilizar en todos los algoritmos será el 3NN. Consiste en considerar, para cada dato, la distancia euclídea entre el dato y todos los demás (excluyéndose él mismo en caso de que estuviéramos preguntando por algún dato dentro del conjunto de entrenamiento (leave one out)) y quedarnos con las clases de los 3 con la distancia más pequeña. La clase del dato en cuestión será de estas tres la que más se repita o, en caso de empate, la clase que corresponda al vecino más cercano.\\
En cada momento la distancia euclídea se calcula teniendo en cuenta las características que están seleccionadas en el momento, por lo que no podemos tener una matriz fija de distancias, hay que ir calculándolas sobre la marcha.

\subsection{Función de evaluación}
La función de evaluación será el rendimiento promedio de un clasificador 3NN en el conjunto de entrenamiento: calcularemos la tasa para cada dato dentro del conjunto de entrenamiento haciendo el leave one out descrito anteriormente y nos quedaremos con la media de las tasas obtenidas. El objetivo será por tanto maximizar esta función.
La tasa se calcula como 100*(nº instancias bien clasificadas / nº total de instancias).\\

El pseudo-código es el siguiente:\\
\begin{lstlisting}
Se obtiene el subconjunto de entrenamiento que se va a tener en cuenta seg|ú|n 
las caracter|í|sticas que se est|é|n considerando.
Para cada dato:
   Se saca del conjunto de entrenamiento (leave one out)
   Se calcula la tasa para este dato
   Se acumula la tasa al resto de tasas
Se divide la acumulaci|ó|n de tasas entre el cardinal del conjunto y 
lo se devuelve.
\end{lstlisting}
La función que hace esto la llamaremos \texttt{CalcularTasa(conjunto, caracteristicas)} donde caracteristicas es la máscara que nos indica cuáles estamos considerando (aquellas que estén a True).

\subsection{Operadores comunes: Generación de vecinos}
Se considerarán como vecinas todas aquellas soluciones que difieran en la pertenencia o no de una única característica (si se diferencian en más de una entonces no es vecina de la considerada). Por ejemplo, las soluciones (True, True, False) y (True, False, False) son vecinas porque se diferencian en una sola característica, la segunda.\\
\texttt{Flip} será el operador de vecino: recibe un vector de máscaras y una posición y cambia esa posición en el vector de máscaras.\\

\newpage

\section{Descripción de los algoritmos}

\subsection{Búsqueda Local}
El algoritmo de búsqueda local implementado ha sido el del primer mejor, es decir, exploramos los vecinos de la solución que tenemos en cada momento y en cuanto obtenemos una mejor nos quedamos con ella y empezamos a generar sus vecinos. Los vecinos se empiezan a generar por una característica aleatoria y a partir de esa característica se van cambiando las demás hasta el final y de nuevo por el principio hasta llegar a la primera que habíamos cambiado. Si no nos quedamos con la solución tenemos que dejar la característica que habíamos cambiado como estaba y si nos la quedamos dejamos de generar vecinos de la solución que teníamos y pasamos a generar vecinos de la nueva solución. El algoritmo para cuando da una pasada entera a los vecinos y no ha encontrado una mejor solución que la que tenía.\\

Veamos el pseudocódigo.\\
\texttt{generarSecuencia(tamaño)} será la función que empieza desde un número aleatorio (menor que el número de características) y de ahí en orden hasta el total de características y empieza de nuevo hasta el número que se ha generado al principio. Esta función nos dará el orden el que recorreremos los vecinos y la llamaremos cada vez que actualicemos la solución.\\
Al algoritmo le pasamos como parámetros los datos de entrenamiento y las clases de los datos.
\begin{lstlisting}
Empezar
   caract = soluci|ó|n aleatoria inicial
   tasa actual = calcularTasa(datos, caract)
   
   while haya mejora en el vecindario y haya menos de 15000 evaluaciones
    hacer:
      
      posiciones = generarSecuencia(tam(caract))
      for j en posiciones:
         Flip(caract, j)  # Generamos vecino
         calcularTasa(caract)
         if la tasa del vecino es mejor que la actual:
            actualizamos la tasa actual
            vuelta_completa = False # hemos encontrado mejora antes de
             # generar todos los vecinos)
            break, salimos del bucle de dentro y empezamos a 
             generar vecinos de la nueva soluci|ó|n
         
         else, volvemos a la antigua soluci|ó|n:
            Flip(caract, j)
            
         if vuelta_completa:
            no ha habido mejora en el vecindario, break
         else:
            vuelta_completa = True
         
         if hemos superado el numero de evaluaciones
            break y salimos del bucle de dentro
   
   Devuelve caract y tasa actual
Fin
   
\end{lstlisting}

\subsection{Enfriamiento Simulado}
El esquema de enfriamiento que he utilizado ha sido el esquema de Cauchy modificado, en el que la temperatura inicial y la modificación en cada iteración del algoritmo se realiza de la siguiente forma:
\begin{lstlisting}
Tf = 0.001, fi = 0.3, mu = 0.3
Tini = (mu*tasa_inicial)/(-np.log(fi))
beta = (Tini - Tf)/(M*Tini*Tf) #M depende del n|ú|mero de vecinos a generar 
   			       #y del n|ú|mero de evaluaciones
Tk = Tk/(1+beta*Tk) 	#Al final del bucle de enfriamiento
\end{lstlisting}

El algoritmo de enfriamiento simulado intenta mejorar el problema de la búsqueda local de quedarse en el primer óptimo local que encuentra, aceptando para ello soluciones que pueden ser peores que la actual bajo algunas circunstancias, en concreto, la temperatura. Al iniciar el algoritmo la temperatura es más alta y se va reduciendo con las iteraciones: la idea es aceptar más soluciones peores que la actual cuanto mayor es la temperatura, es decir, al inicio del algoritmo, e ir aceptando menos progresivamente según la temperatura va bajando. Devolvemos la mejor solución encontrada en general a lo largo del algoritmo. Necesitamos para ello guardar la mejor solución junto con su tasa y la solución actual junto con su tasa.\\
En cada enfriamiento generamos vecinos hasta un máximo de forma aleatoria y nos la quedamos si es mejor que la actual o supera la probabilidad de aceptar una solución peor.\\

Con esto, el pseudocódigo es el siguiente:
\begin{lstlisting}
Empezar
caract = soluci|ó|n aleatoria inicial
mejor solucion = caract
mejor tasa = calcularTasa(datos, caract)
Se calculan los par|á|metros para la temperatura y los vecinos totales
while haya |é|xitos en el enfriamiento actual y Tk>Tf y haya menos de 15000 evals:
   while no se haya generado el m|á|ximo de vecinos ni se sobrepasen los |é|xitos:
      pos = num aleatorio entre 0 y n #n num de caracter|í|sticas
      caract = Flip(caract, pos)
      nueva tasa = calcularTasa(datos, caract)
      Se aumenta num vecinos y num evaluaciones
      delta = nueva tasa - tasa actual
      if delta!=0 y (delta>0 o U(0,1)<=exp(delta/Tk)):
         #Se ha aceptado el vecino
         Se actualiza la tasa actual
         Se aumenta el n|ú|mero de |é|xitos
         if tasa actual > mejor tasa:
            Se actualizan la mejor soluci|ó|n y la mejor tasa
      
      else:
         #No se ha aceptado el vecino, volvemos a la soluci|ó|n anterior
         Flip(caract, pos)
         
      Comprobamos si hemos pasado el n|ú|mero de evaluaciones
   Fin del while interior
   
   Comprobamos si ha habido |é|xitos en este enfriamiento
   Ponemos el n|ú|mero de |é|xitos a 0
   Ponemos el n|ú|mero de vecinos a 0
      
   Tk = Tk/(1+beta*Tk) #Actualizamos temperatura
   Fin del while exterior
      
Devolvemos mejor soluci|ó|n y mejor tasa
Fin
\end{lstlisting}

$U(0,1)$ es un número aleatorio extraído de la distribución uniforme en el intervalo $[0,1]$.\\

He tenido que poner como condición adicional en el \texttt{if} para aceptar una solución que \texttt{delta!=0} porque si las tasas son iguales y la diferencia es cero entonces la exponencial de $delta/Tk$ es siempre 1 al ser delta 0 y un número entre 0 y 1 es siempre menor o igual que 1, con lo que siempre cogía como éxito una solución con la misma tasa, haciendo que el algoritmo tardara mucho más al haber siempre éxitos en el enfriamiento actual, cuando en realidad no se estaba moviendo ni hacia arriba ni hacia abajo.

\subsection{Búsqueda Tabú Básica}
En la búsqueda tabú también buscamos salir de óptimos locales con unos mecanismos distintos a los del enfriamiento simulado. En este caso vamos a tener almacenada una lista tabú que nos guarda las posiciones de las características que hemos cambiado en los $n/3$ movimientos anteriores, donde $n$ es el número de características.  La lista tiene por tanto $n/3$ posiciones y está implementada como una lista circular, de forma que tenemos un índice marcando la última posición añadida y el siguiente a ese índice módulo $n/3$ es el próximo que tenemos que modificar. Se guardan estas posiciones para que no se pueda volver a cambiar una característica en esas posiciones para no volver hacia soluciones que habíamos considerado peores, a no ser que modificar una característica en la lista tabú mejore a la mejor solución que tenemos en ese momento. El método de búsqueda en este caso es la generación de 30 vecinos aleatorios de la solución actual y quedarnos con el mejor de ellos, sea mejor o no que el que teníamos. Guardamos siempre la mejor solución, que será la que se devolverá.\\
Con esto, el pseudocódigo es el siguiente:

\begin{lstlisting}
Empezar
caract = soluci|ó|n aleatoria inicial
mejor tasa = calcularTasa(datos, caract)
mejor soluci|ó|n = caract
lista tab|ú| = vector de longitud n/3 inicializado a -1
plista = -1		#indica la |ú|ltima posici|ó|n modificada en la lista

while num evaluaciones < 15000:
   tasa actual = 0
   mejor pos = -1
   posiciones = 30 vecinos generados aleatoriamente
   for j en posiciones:
      caract = Flip(caract, j) 	#Generamos vecino
      nueva tasa = calcularTasa(datos, caract)
      Se aumenta el n|ú|mero de evaluaciones
      if j esta en la lista tab|ú|:
         if nueva tasa > mejor tasa y nueva tasa > tasa actual:
            Se actualiza la tasa actual con la nueva
            Se cambia j como mejor posici|ó|n
      elif nueva tasa > tasa actual:
         Se actualiza la tasa actual con la nueva
         Se cambia j como mejor posici|ó|n
      # Volvemos a la soluci|ó|n que ten|í|amos para seguir generando vecinos
      caract = Flip(datos, caract)
   
   Fin for
   # Nos quedamos con el mejor vecino
   caract = Flip(caract, mejor pos)
   plista = (plista+1)mod(n/3)
   lista tabu[plista] = mejor pos
   
   if tasa actual > mejor tasa:
      Se actualiza la mejor tasa y la mejor soluci|ó|n.

   Fin while
   
   Devuelve mejor soluci|ó|n y mejor pos
Fin

\end{lstlisting}

\subsection{Búsqueda Tabú Extendida}
La búsqueda tabú extendida utiliza el mismo mecanismo de memoria a corto plazo que la tabú sencilla, la lista tabú, aunque le iremos cambiando el tamaño. Incorpora además una memoria a largo plazo en la que se guarda el número de veces que se selecciona una característica para obtener la frecuencia de cada característica y si en algún momento se llega a 10 iteraciones sin mejora en la mejor solución, se reinicializa la búsqueda desde una nueva solución que tendrá probabilidad 0.25 de ser aleatoria, probabilidad 0.25 de ser la mejor solución y probabilidad 0.5 de ser generada de forma aleatoria utilizando el vector de frecuencias.\\

El pseudocódigo es el mismo que el de la búsqueda tabú sencilla con una modificación al final:

\begin{lstlisting}
Empezar
... (igual que antes)
frec = vector de frecuencias de longitud n (num caracter|í|sticas)
 inicializado a cero

while num evaluaciones < 15000:
   ... (igual que antes)
   for j en posiciones:
      ... (igual que antes)
   Fin for
   
   # Nos quedamos con el mejor vecino
   caract = Flip(caract, mejor pos)
   plista = (plista+1)mod(n/3)
   lista tabu[plista] = mejor pos
   Se aumenta frec[mejor pos] en 1   
   
   if tasa actual > mejor tasa:
      Se actualiza la mejor tasa y la mejor soluci|ó|n.
      no_mejora = 0
   else:
      no_mejora += 1
   
   if no_mejora llega a 10:
      no_mejora = 0
      Se elige un numero entre 3 en el que los dos primeros tienen
       probabilidad 0.25 de aparecer y el tercero tiene 0.5
          if sale el primero:
             caract = soluci|ó|n aleatoria
          elif sale el segundo:
             caract = mejor soluci|ó|n
          else:
             u = n|ú|mero uniforme entre 0 y 1
             for i de 0 hasta n:  # n num caracter|í|sticas
                 if u < 1 - frec[i]:
                    se pone la caracter|í|stica i a True
                 else
                    se pone a False
      u = n|ú|mero uniforme entre 0 y 1
      if u < 0.5
         Se aumenta al doble la lista tab|ú|
      else:
         Se disminuye a la mitad la lista tab|ú|

   Fin while
   
   Devuelve mejor soluci|ó|n y mejor pos
Fin

\end{lstlisting}

\newpage

\section{Breve descripción del algoritmo de comparación}
El algoritmo de comparación seleccionado ha sido el greedy Sequential Forward Selection (SFS), que parte de una solución inicial en la que no hay ninguna característica seleccionada y se va quedando en cada iteración con la característica con la que se obtiene la mejor tasa. El algoritmo no para mientras se encuentre mejora añadiendo alguna característica.\\
He implementado una función que me devuelve, para una máscara determinada, la característica más prometedora que se puede obtener, cuyo pseudocódigo es el siguiente:
\begin{lstlisting}
Empezar caractMasPrometedora(mascara):
   posiciones = posiciones que no est|é|n seleccionadas de la mascara
   for i en posiciones:
      mascara[i] = True
      Se calcula la tasa con la nueva caracter|í|stica a|ñ|adida
      if nueva tasa > mejor tasa:
         Se actualiza la mejor tasa
         Se actualiza la mejor posici|ó|n
      Fin for
   Devuelve mejor tasa y mejor pos
Fin
\end{lstlisting}

Con esto, el pseudocódigo del algoritmo SFS es:
\begin{lstlisting}
Empezar
caract = soluci|ó|n inicial inicializada a False
tasa actual = 0
mejora = True
while haya mejora:
   Se calcula la tasa y la mejor posici|ó|n con caractMasPrometedora
   if nueva tasa > tasa actual:
      Se actualiza la tasa actual
      Se pone a True la caracter|í|stica en mejor posici|ó|n
   else:
      mejora = False	#No ha habido mejora: paramos
      
   Fin while
   
   Devuelve caract y tasa
Fin

\end{lstlisting}

\newpage

\section{Procedimiento considerado para desarrollar la práctica}


\section{Experimentos y análisis de resultados}
\subsection{Descripción de los casos del problema empleados}

\subsection{Resultados}
\begin{table}[H]
\centering
\caption{Resultados SFS}
\label{Resultados SFS}
\resizebox{\textwidth}{!}{\begin{tabular}{|c|cccc|cccc|cccc|}
\hline
              &                                                                               & Wdbc                                                                         &                              &         &                                                                               & \begin{tabular}[c]{@{}c@{}}Movement\\ Libras\end{tabular}                    &                              &          &                                                                               & Arrhythmia                                                                   &                              &          \\ \cline{2-13} 
              & \multicolumn{1}{c|}{\begin{tabular}[c]{@{}c@{}}\%\_clas\\ train\end{tabular}} & \multicolumn{1}{c|}{\begin{tabular}[c]{@{}c@{}}\%\_clas\\ test\end{tabular}} & \multicolumn{1}{c|}{\%\_red} & T (s)   & \multicolumn{1}{c|}{\begin{tabular}[c]{@{}c@{}}\%\_clas\\ train\end{tabular}} & \multicolumn{1}{c|}{\begin{tabular}[c]{@{}c@{}}\%\_clas\\ test\end{tabular}} & \multicolumn{1}{c|}{\%\_red} & T (s)    & \multicolumn{1}{c|}{\begin{tabular}[c]{@{}c@{}}\%\_clas\\ train\end{tabular}} & \multicolumn{1}{c|}{\begin{tabular}[c]{@{}c@{}}\%\_clas\\ test\end{tabular}} & \multicolumn{1}{c|}{\%\_red} & T (s)    \\ \hline
Partición 1-1 & \multicolumn{1}{c|}{98,9437}                                                  & \multicolumn{1}{c|}{93,3333}                                                 & \multicolumn{1}{c|}{83,3333} & 47,7583 & \multicolumn{1}{c|}{65,5556}                                                  & \multicolumn{1}{c|}{64,4444}                                                 & \multicolumn{1}{c|}{92,2222} & 128,1156 & \multicolumn{1}{c|}{81,7708}                                                  & \multicolumn{1}{c|}{70,1031}                                                 & \multicolumn{1}{c|}{97,1223} & 474,7185 \\ \hline
Partición 1-2 & \multicolumn{1}{c|}{95,7895}                                                  & \multicolumn{1}{c|}{94,0141}                                                 & \multicolumn{1}{c|}{83,3333} & 47,5713 & \multicolumn{1}{c|}{73,8889}                                                  & \multicolumn{1}{c|}{70,0000}                                                 & \multicolumn{1}{c|}{91,1111} & 141,5071 & \multicolumn{1}{c|}{80,9278}                                                  & \multicolumn{1}{c|}{78,6458}                                                 & \multicolumn{1}{c|}{97,1223} & 474,0171 \\ \hline
Partición 2-1 & \multicolumn{1}{c|}{95,4225}                                                  & \multicolumn{1}{c|}{93,3333}                                                 & \multicolumn{1}{c|}{90,0000} & 32,6485 & \multicolumn{1}{c|}{70,0000}                                                  & \multicolumn{1}{c|}{70,5556}                                                 & \multicolumn{1}{c|}{91,1111} & 141,5607 & \multicolumn{1}{c|}{78,6458}                                                  & \multicolumn{1}{c|}{70,1031}                                                 & \multicolumn{1}{c|}{97,4820} & 417,5319 \\ \hline
Partición 2-2 & \multicolumn{1}{c|}{97,5439}                                                  & \multicolumn{1}{c|}{94,3662}                                                 & \multicolumn{1}{c|}{90,0000} & 32,8499 & \multicolumn{1}{c|}{67,2222}                                                  & \multicolumn{1}{c|}{66,1111}                                                 & \multicolumn{1}{c|}{93,3333} & 110,7466 & \multicolumn{1}{c|}{85,0515}                                                  & \multicolumn{1}{c|}{70,8333}                                                 & \multicolumn{1}{c|}{96,4029} & 582,5531 \\ \hline
Partición 3-1 & \multicolumn{1}{c|}{97,8873}                                                  & \multicolumn{1}{c|}{96,4912}                                                 & \multicolumn{1}{c|}{76,6667} & 61,5079 & \multicolumn{1}{c|}{71,1111}                                                  & \multicolumn{1}{c|}{70,0000}                                                 & \multicolumn{1}{c|}{90,0000} & 159,5794 & \multicolumn{1}{c|}{75,0000}                                                  & \multicolumn{1}{c|}{69,0722}                                                 & \multicolumn{1}{c|}{98,9209} & 209,9304 \\ \hline
Partición 3-2 & \multicolumn{1}{c|}{97,5439}                                                  & \multicolumn{1}{c|}{93,6620}                                                 & \multicolumn{1}{c|}{86,6667} & 40,9398 & \multicolumn{1}{c|}{66,6667}                                                  & \multicolumn{1}{c|}{70,0000}                                                 & \multicolumn{1}{c|}{92,2222} & 127,1383 & \multicolumn{1}{c|}{78,8660}                                                  & \multicolumn{1}{c|}{71,3542}                                                 & \multicolumn{1}{c|}{97,8417} & 365,9823 \\ \hline
Partición 4-1 & \multicolumn{1}{c|}{96,8310}                                                  & \multicolumn{1}{c|}{97,8947}                                                 & \multicolumn{1}{c|}{86,6667} & 39,8884 & \multicolumn{1}{c|}{67,7778}                                                  & \multicolumn{1}{c|}{70,5556}                                                 & \multicolumn{1}{c|}{92,2222} & 128,9092 & \multicolumn{1}{c|}{82,8125}                                                  & \multicolumn{1}{c|}{73,7113}                                                 & \multicolumn{1}{c|}{96,7626} & 526,7995 \\ \hline
Partición 4-2 & \multicolumn{1}{c|}{97,5439}                                                  & \multicolumn{1}{c|}{95,4225}                                                 & \multicolumn{1}{c|}{83,3333} & 46,9100 & \multicolumn{1}{c|}{78,3333}                                                  & \multicolumn{1}{c|}{62,2222}                                                 & \multicolumn{1}{c|}{90,0000} & 157,2638 & \multicolumn{1}{c|}{85,0515}                                                  & \multicolumn{1}{c|}{73,4375}                                                 & \multicolumn{1}{c|}{96,0432} & 632,3626 \\ \hline
Partición 5-1 & \multicolumn{1}{c|}{97,5352}                                                  & \multicolumn{1}{c|}{95,7895}                                                 & \multicolumn{1}{c|}{90,0000} & 32,3780 & \multicolumn{1}{c|}{80,0000}                                                  & \multicolumn{1}{c|}{62,2222}                                                 & \multicolumn{1}{c|}{86,6667} & 204,0013 & \multicolumn{1}{c|}{82,2917}                                                  & \multicolumn{1}{c|}{68,0412}                                                 & \multicolumn{1}{c|}{97,4820} & 424,4544 \\ \hline
Partición 5-2 & \multicolumn{1}{c|}{96,1404}                                                  & \multicolumn{1}{c|}{94,7183}                                                 & \multicolumn{1}{c|}{83,3333} & 47,9195 & \multicolumn{1}{c|}{77,7778}                                                  & \multicolumn{1}{c|}{68,8889}                                                 & \multicolumn{1}{c|}{90,0000} & 159,4742 & \multicolumn{1}{c|}{78,8660}                                                  & \multicolumn{1}{c|}{72,3958}                                                 & \multicolumn{1}{c|}{97,1223} & 485,9460 \\ \hline
Media         & \multicolumn{1}{c|}{97,1181}                                                  & \multicolumn{1}{c|}{94,9025}                                                 & \multicolumn{1}{c|}{85,3333} & 43,0372 & \multicolumn{1}{c|}{71,8333}                                                  & \multicolumn{1}{c|}{67,5000}                                                 & \multicolumn{1}{c|}{90,8889} & 145,8296 & \multicolumn{1}{c|}{80,9284}                                                  & \multicolumn{1}{c|}{71,7698}                                                 & \multicolumn{1}{c|}{97,2302} & 459,4296 \\ \hline
\end{tabular}}
\end{table}

\begin{table}[H]
\centering
\caption{Resultados BL}
\label{Resultados BL}
\resizebox{\textwidth}{!}{\begin{tabular}{|c|cccc|cccc|cccc|}
\hline
              &                                                                               & Wdbc                                                                         &                              &         &                                                                               & \begin{tabular}[c]{@{}c@{}}Movement\\ Libras\end{tabular}                    &                              &         &                                                                               & Arrhythmia                                                                   &                              &         \\ \cline{2-13} 
              & \multicolumn{1}{c|}{\begin{tabular}[c]{@{}c@{}}\%\_clas\\ train\end{tabular}} & \multicolumn{1}{c|}{\begin{tabular}[c]{@{}c@{}}\%\_clas\\ test\end{tabular}} & \multicolumn{1}{c|}{\%\_red} & T (s)   & \multicolumn{1}{c|}{\begin{tabular}[c]{@{}c@{}}\%\_clas\\ train\end{tabular}} & \multicolumn{1}{c|}{\begin{tabular}[c]{@{}c@{}}\%\_clas\\ test\end{tabular}} & \multicolumn{1}{c|}{\%\_red} & T (s)   & \multicolumn{1}{c|}{\begin{tabular}[c]{@{}c@{}}\%\_clas\\ train\end{tabular}} & \multicolumn{1}{c|}{\begin{tabular}[c]{@{}c@{}}\%\_clas\\ test\end{tabular}} & \multicolumn{1}{c|}{\%\_red} & T (s)   \\ \hline
Partición 1-1 & \multicolumn{1}{c|}{97,5352}                                                  & \multicolumn{1}{c|}{93,6842}                                                 & \multicolumn{1}{c|}{46,6667} & 3,7190  & \multicolumn{1}{c|}{62,2222}                                                  & \multicolumn{1}{c|}{69,4444}                                                 & \multicolumn{1}{c|}{52,2222} & 8,8476  & \multicolumn{1}{c|}{68,2292}                                                  & \multicolumn{1}{c|}{63,4021}                                                 & \multicolumn{1}{c|}{50,3597} & 9,8135  \\ \hline
Partición 1-2 & \multicolumn{1}{c|}{94,3860}                                                  & \multicolumn{1}{c|}{95,4225}                                                 & \multicolumn{1}{c|}{46,6667} & 1,2674  & \multicolumn{1}{c|}{69,4444}                                                  & \multicolumn{1}{c|}{76,1111}                                                 & \multicolumn{1}{c|}{46,6667} & 12,3067 & \multicolumn{1}{c|}{67,0103}                                                  & \multicolumn{1}{c|}{63,5417}                                                 & \multicolumn{1}{c|}{45,3237} & 32,3330 \\ \hline
Partición 2-1 & \multicolumn{1}{c|}{95,7746}                                                  & \multicolumn{1}{c|}{96,8421}                                                 & \multicolumn{1}{c|}{63,3333} & 3,3681  & \multicolumn{1}{c|}{67,7778}                                                  & \multicolumn{1}{c|}{74,4444}                                                 & \multicolumn{1}{c|}{57,7778} & 1,5256  & \multicolumn{1}{c|}{65,6250}                                                  & \multicolumn{1}{c|}{61,3402}                                                 & \multicolumn{1}{c|}{56,1151} & 65,7732 \\ \hline
Partición 2-2 & \multicolumn{1}{c|}{96,4912}                                                  & \multicolumn{1}{c|}{92,9577}                                                 & \multicolumn{1}{c|}{46,6667} & 7,1604  & \multicolumn{1}{c|}{65,5556}                                                  & \multicolumn{1}{c|}{65,0000}                                                 & \multicolumn{1}{c|}{56,6667} & 17,8602 & \multicolumn{1}{c|}{68,0412}                                                  & \multicolumn{1}{c|}{64,5833}                                                 & \multicolumn{1}{c|}{46,4029} & 51,7505 \\ \hline
Partición 3-1 & \multicolumn{1}{c|}{96,1268}                                                  & \multicolumn{1}{c|}{96,1404}                                                 & \multicolumn{1}{c|}{33,3333} & 11,9843 & \multicolumn{1}{c|}{67,7778}                                                  & \multicolumn{1}{c|}{70,0000}                                                 & \multicolumn{1}{c|}{56,6667} & 10,5577 & \multicolumn{1}{c|}{66,6667}                                                  & \multicolumn{1}{c|}{63,9175}                                                 & \multicolumn{1}{c|}{46,4029} & 32,6144 \\ \hline
Partición 3-2 & \multicolumn{1}{c|}{96,4912}                                                  & \multicolumn{1}{c|}{97,1831}                                                 & \multicolumn{1}{c|}{43,3333} & 11,0846 & \multicolumn{1}{c|}{66,6667}                                                  & \multicolumn{1}{c|}{70,5556}                                                 & \multicolumn{1}{c|}{50,0000} & 7,8249  & \multicolumn{1}{c|}{67,5258}                                                  & \multicolumn{1}{c|}{65,6250}                                                 & \multicolumn{1}{c|}{52,5180} & 36,3594 \\ \hline
Partición 4-1 & \multicolumn{1}{c|}{96,1268}                                                  & \multicolumn{1}{c|}{95,4386}                                                 & \multicolumn{1}{c|}{40,0000} & 5,8840  & \multicolumn{1}{c|}{59,4444}                                                  & \multicolumn{1}{c|}{70,5556}                                                 & \multicolumn{1}{c|}{53,3333} & 7,8953  & \multicolumn{1}{c|}{67,1875}                                                  & \multicolumn{1}{c|}{62,3711}                                                 & \multicolumn{1}{c|}{43,8849} & 16,5154 \\ \hline
Partición 4-2 & \multicolumn{1}{c|}{95,7895}                                                  & \multicolumn{1}{c|}{94,3662}                                                 & \multicolumn{1}{c|}{63,3333} & 10,4619 & \multicolumn{1}{c|}{72,2222}                                                  & \multicolumn{1}{c|}{61,6667}                                                 & \multicolumn{1}{c|}{44,4444} & 7,9087  & \multicolumn{1}{c|}{65,9794}                                                  & \multicolumn{1}{c|}{66,6667}                                                 & \multicolumn{1}{c|}{52,8777} & 4,9554  \\ \hline
Partición 5-1 & \multicolumn{1}{c|}{96,1268}                                                  & \multicolumn{1}{c|}{96,8421}                                                 & \multicolumn{1}{c|}{46,6667} & 9,4036  & \multicolumn{1}{c|}{71,1111}                                                  & \multicolumn{1}{c|}{67,7778}                                                 & \multicolumn{1}{c|}{53,3333} & 8,5307  & \multicolumn{1}{c|}{65,1042}                                                  & \multicolumn{1}{c|}{64,9485}                                                 & \multicolumn{1}{c|}{48,9209} & 4,5913  \\ \hline
Partición 5-2 & \multicolumn{1}{c|}{97,5439}                                                  & \multicolumn{1}{c|}{95,7746}                                                 & \multicolumn{1}{c|}{50,0000} & 8,3311  & \multicolumn{1}{c|}{71,1111}                                                  & \multicolumn{1}{c|}{68,3333}                                                 & \multicolumn{1}{c|}{51,1111} & 11,8468 & \multicolumn{1}{c|}{63,9175}                                                  & \multicolumn{1}{c|}{63,0208}                                                 & \multicolumn{1}{c|}{46,0432} & 42,4668 \\ \hline
Media         & \multicolumn{1}{c|}{96,2392}                                                  & \multicolumn{1}{c|}{95,4652}                                                 & \multicolumn{1}{c|}{48,0000} & 7,2665  & \multicolumn{1}{c|}{67,3333}                                                  & \multicolumn{1}{c|}{69,3889}                                                 & \multicolumn{1}{c|}{52,2222} & 9,5104  & \multicolumn{1}{c|}{66,5287}                                                  & \multicolumn{1}{c|}{63,9417}                                                 & \multicolumn{1}{c|}{48,8849} & 29,7173 \\ \hline
\end{tabular}}
\end{table}

\begin{table}[H]
\centering
\caption{Resultados ES}
\label{Resultados ES}
\resizebox{\textwidth}{!}{\begin{tabular}{|c|cccc|cccc|cccc|}
\hline
              &                                                                               & Wdbc                                                                         &                              &          &                                                                               & \begin{tabular}[c]{@{}c@{}}Movement\\ Libras\end{tabular}                    &                              &          &                                                                               & Arrhythmia                                                                   &                              &           \\ \cline{2-13} 
              & \multicolumn{1}{c|}{\begin{tabular}[c]{@{}c@{}}\%\_clas\\ train\end{tabular}} & \multicolumn{1}{c|}{\begin{tabular}[c]{@{}c@{}}\%\_clas\\ test\end{tabular}} & \multicolumn{1}{c|}{\%\_red} & T (s)    & \multicolumn{1}{c|}{\begin{tabular}[c]{@{}c@{}}\%\_clas\\ train\end{tabular}} & \multicolumn{1}{c|}{\begin{tabular}[c]{@{}c@{}}\%\_clas\\ test\end{tabular}} & \multicolumn{1}{c|}{\%\_red} & T (s)    & \multicolumn{1}{c|}{\begin{tabular}[c]{@{}c@{}}\%\_clas\\ train\end{tabular}} & \multicolumn{1}{c|}{\begin{tabular}[c]{@{}c@{}}\%\_clas\\ test\end{tabular}} & \multicolumn{1}{c|}{\%\_red} & T (s)     \\ \hline
Partición 1-1 & \multicolumn{1}{c|}{98,9437}                                                  & \multicolumn{1}{c|}{95,0877}                                                 & \multicolumn{1}{c|}{43,3333} & 190,6666 & \multicolumn{1}{c|}{61,1111}                                                  & \multicolumn{1}{c|}{68,3333}                                                 & \multicolumn{1}{c|}{51,1111} & 383,7357 & \multicolumn{1}{c|}{73,9583}                                                  & \multicolumn{1}{c|}{62,3711}                                                 & \multicolumn{1}{c|}{49,6403} & 1641,5006 \\ \hline
Partición 1-2 & \multicolumn{1}{c|}{95,7895}                                                  & \multicolumn{1}{c|}{96,1268}                                                 & \multicolumn{1}{c|}{46,6667} & 186,5556 & \multicolumn{1}{c|}{70,0000}                                                  & \multicolumn{1}{c|}{73,3333}                                                 & \multicolumn{1}{c|}{55,5556} & 381,7689 & \multicolumn{1}{c|}{70,1031}                                                  & \multicolumn{1}{c|}{62,5000}                                                 & \multicolumn{1}{c|}{50,0000} & 1663,5517 \\ \hline
Partición 2-1 & \multicolumn{1}{c|}{97,5352}                                                  & \multicolumn{1}{c|}{95,7895}                                                 & \multicolumn{1}{c|}{50,0000} & 185,7704 & \multicolumn{1}{c|}{70,5556}                                                  & \multicolumn{1}{c|}{75,5556}                                                 & \multicolumn{1}{c|}{62,2222} & 383,1403 & \multicolumn{1}{c|}{66,1458}                                                  & \multicolumn{1}{c|}{61,8557}                                                 & \multicolumn{1}{c|}{49,2806} & 1571,5591 \\ \hline
Partición 2-2 & \multicolumn{1}{c|}{98,2456}                                                  & \multicolumn{1}{c|}{94,0141}                                                 & \multicolumn{1}{c|}{53,3333} & 188,5999 & \multicolumn{1}{c|}{66,6667}                                                  & \multicolumn{1}{c|}{62,7778}                                                 & \multicolumn{1}{c|}{44,4444} & 390,1831 & \multicolumn{1}{c|}{70,1031}                                                  & \multicolumn{1}{c|}{63,5417}                                                 & \multicolumn{1}{c|}{51,0791} & 1562,8782 \\ \hline
Partición 3-1 & \multicolumn{1}{c|}{97,1831}                                                  & \multicolumn{1}{c|}{95,0877}                                                 & \multicolumn{1}{c|}{60,0000} & 185,9046 & \multicolumn{1}{c|}{66,6667}                                                  & \multicolumn{1}{c|}{72,2222}                                                 & \multicolumn{1}{c|}{55,5556} & 384,5502 & \multicolumn{1}{c|}{69,2708}                                                  & \multicolumn{1}{c|}{65,4639}                                                 & \multicolumn{1}{c|}{47,4820} & 1593,5539 \\ \hline
Partición 3-2 & \multicolumn{1}{c|}{97,8947}                                                  & \multicolumn{1}{c|}{94,7183}                                                 & \multicolumn{1}{c|}{46,6667} & 188,8094 & \multicolumn{1}{c|}{68,8889}                                                  & \multicolumn{1}{c|}{73,3333}                                                 & \multicolumn{1}{c|}{53,3333} & 374,4300 & \multicolumn{1}{c|}{70,6186}                                                  & \multicolumn{1}{c|}{66,6667}                                                 & \multicolumn{1}{c|}{53,5971} & 1536,6501 \\ \hline
Partición 4-1 & \multicolumn{1}{c|}{97,1831}                                                  & \multicolumn{1}{c|}{96,8421}                                                 & \multicolumn{1}{c|}{53,3333} & 197,0699 & \multicolumn{1}{c|}{62,7778}                                                  & \multicolumn{1}{c|}{76,1111}                                                 & \multicolumn{1}{c|}{52,2222} & 379,0105 & \multicolumn{1}{c|}{70,8333}                                                  & \multicolumn{1}{c|}{64,9485}                                                 & \multicolumn{1}{c|}{50,0000} & 1532,1212 \\ \hline
Partición 4-2 & \multicolumn{1}{c|}{96,8421}                                                  & \multicolumn{1}{c|}{94,0141}                                                 & \multicolumn{1}{c|}{36,6667} & 210,1286 & \multicolumn{1}{c|}{73,3333}                                                  & \multicolumn{1}{c|}{59,4444}                                                 & \multicolumn{1}{c|}{55,5556} & 384,5273 & \multicolumn{1}{c|}{69,5876}                                                  & \multicolumn{1}{c|}{66,1458}                                                 & \multicolumn{1}{c|}{47,1223} & 1599,9626 \\ \hline
Partición 5-1 & \multicolumn{1}{c|}{96,8310}                                                  & \multicolumn{1}{c|}{96,8421}                                                 & \multicolumn{1}{c|}{50,0000} & 195,4052 & \multicolumn{1}{c|}{71,1111}                                                  & \multicolumn{1}{c|}{66,6667}                                                 & \multicolumn{1}{c|}{41,1111} & 391,5431 & \multicolumn{1}{c|}{71,8750}                                                  & \multicolumn{1}{c|}{64,9485}                                                 & \multicolumn{1}{c|}{48,5612} & 1600,2462 \\ \hline
Partición 5-2 & \multicolumn{1}{c|}{97,5439}                                                  & \multicolumn{1}{c|}{96,4789}                                                 & \multicolumn{1}{c|}{53,3333} & 189,5962 & \multicolumn{1}{c|}{72,2222}                                                  & \multicolumn{1}{c|}{70,0000}                                                 & \multicolumn{1}{c|}{48,8889} & 387,5911 & \multicolumn{1}{c|}{68,0412}                                                  & \multicolumn{1}{c|}{68,7500}                                                 & \multicolumn{1}{c|}{50,3597} & 1555,9471 \\ \hline
Media         & \multicolumn{1}{c|}{97,3992}                                                  & \multicolumn{1}{c|}{95,5001}                                                 & \multicolumn{1}{c|}{49,3333} & 191,8507 & \multicolumn{1}{c|}{68,3333}                                                  & \multicolumn{1}{c|}{69,7778}                                                 & \multicolumn{1}{c|}{52,0000} & 384,0480 & \multicolumn{1}{c|}{70,0537}                                                  & \multicolumn{1}{c|}{64,7192}                                                 & \multicolumn{1}{c|}{49,7122} & 1585,7971 \\ \hline
\end{tabular}}
\end{table}


\begin{table}[H]
\centering
\caption{Resultados BT}
\label{Resultados BT}
\resizebox{\textwidth}{!}{\begin{tabular}{|c|cccc|cccc|cccc|}
\hline
              &                                                                               & Wdbc                                                                         &                              &           &                                                                               & \begin{tabular}[c]{@{}c@{}}Movement\\ Libras\end{tabular}                    &                              &           &                                                                               & Arrhythmia                                                                   &                              &           \\ \cline{2-13} 
              & \multicolumn{1}{c|}{\begin{tabular}[c]{@{}c@{}}\%\_clas\\ train\end{tabular}} & \multicolumn{1}{c|}{\begin{tabular}[c]{@{}c@{}}\%\_clas\\ test\end{tabular}} & \multicolumn{1}{c|}{\%\_red} & T (s)     & \multicolumn{1}{c|}{\begin{tabular}[c]{@{}c@{}}\%\_clas\\ train\end{tabular}} & \multicolumn{1}{c|}{\begin{tabular}[c]{@{}c@{}}\%\_clas\\ test\end{tabular}} & \multicolumn{1}{c|}{\%\_red} & T (s)     & \multicolumn{1}{c|}{\begin{tabular}[c]{@{}c@{}}\%\_clas\\ train\end{tabular}} & \multicolumn{1}{c|}{\begin{tabular}[c]{@{}c@{}}\%\_clas\\ test\end{tabular}} & \multicolumn{1}{c|}{\%\_red} & T (s)     \\ \hline
Partición 1-1 & \multicolumn{1}{c|}{99,6479}                                                  & \multicolumn{1}{c|}{94,0351}                                                 & \multicolumn{1}{c|}{53,3333} & 4482,1553 & \multicolumn{1}{c|}{71,1111}                                                  & \multicolumn{1}{c|}{67,7778}                                                 & \multicolumn{1}{c|}{67,7778} & 2792,4891 & \multicolumn{1}{c|}{77,6042}                                                  & \multicolumn{1}{c|}{64,9485}                                                 & \multicolumn{1}{c|}{50,3597} & 3547,2839 \\ \hline
Partición 1-2 & \multicolumn{1}{c|}{97,8947}                                                  & \multicolumn{1}{c|}{96,4789}                                                 & \multicolumn{1}{c|}{40,0000} & 4525,3261 & \multicolumn{1}{c|}{77,7778}                                                  & \multicolumn{1}{c|}{71,1111}                                                 & \multicolumn{1}{c|}{50,0000} & 2811,5240 & \multicolumn{1}{c|}{72,1649}                                                  & \multicolumn{1}{c|}{61,4583}                                                 & \multicolumn{1}{c|}{54,3165} & 3689,6839 \\ \hline
Partición 2-1 & \multicolumn{1}{c|}{98,2394}                                                  & \multicolumn{1}{c|}{96,8421}                                                 & \multicolumn{1}{c|}{40,0000} & 4414,8685 & \multicolumn{1}{c|}{77,7778}                                                  & \multicolumn{1}{c|}{75,5556}                                                 & \multicolumn{1}{c|}{51,1111} & 2816,0023 & \multicolumn{1}{c|}{69,7917}                                                  & \multicolumn{1}{c|}{62,8866}                                                 & \multicolumn{1}{c|}{56,8345} & 3547,7961 \\ \hline
Partición 2-2 & \multicolumn{1}{c|}{99,6491}                                                  & \multicolumn{1}{c|}{94,3662}                                                 & \multicolumn{1}{c|}{50,0000} & 4387,6807 & \multicolumn{1}{c|}{72,7778}                                                  & \multicolumn{1}{c|}{65,5556}                                                 & \multicolumn{1}{c|}{57,7778} & 2793,7549 & \multicolumn{1}{c|}{78,8660}                                                  & \multicolumn{1}{c|}{65,1042}                                                 & \multicolumn{1}{c|}{56,4748} & 3540,1120 \\ \hline
Partición 3-1 & \multicolumn{1}{c|}{98,9437}                                                  & \multicolumn{1}{c|}{94,3860}                                                 & \multicolumn{1}{c|}{56,6667} & 4435,5301 & \multicolumn{1}{c|}{75,5556}                                                  & \multicolumn{1}{c|}{72,7778}                                                 & \multicolumn{1}{c|}{58,8889} & 2804,0871 & \multicolumn{1}{c|}{75,0000}                                                  & \multicolumn{1}{c|}{65,9794}                                                 & \multicolumn{1}{c|}{60,0719} & 3567,9060 \\ \hline
Partición 3-2 & \multicolumn{1}{c|}{98,5965}                                                  & \multicolumn{1}{c|}{95,7746}                                                 & \multicolumn{1}{c|}{63,3333} & 4402,7661 & \multicolumn{1}{c|}{76,6667}                                                  & \multicolumn{1}{c|}{70,5556}                                                 & \multicolumn{1}{c|}{63,3333} & 2784,8322 & \multicolumn{1}{c|}{75,7732}                                                  & \multicolumn{1}{c|}{67,7083}                                                 & \multicolumn{1}{c|}{56,4748} & 3516,0451 \\ \hline
Partición 4-1 & \multicolumn{1}{c|}{98,2394}                                                  & \multicolumn{1}{c|}{96,1404}                                                 & \multicolumn{1}{c|}{50,0000} & 4398,5370 & \multicolumn{1}{c|}{72,2222}                                                  & \multicolumn{1}{c|}{73,8889}                                                 & \multicolumn{1}{c|}{56,6667} & 2817,5327 & \multicolumn{1}{c|}{76,5625}                                                  & \multicolumn{1}{c|}{65,9794}                                                 & \multicolumn{1}{c|}{53,2374} & 3498,7134 \\ \hline
Partición 4-2 & \multicolumn{1}{c|}{98,5965}                                                  & \multicolumn{1}{c|}{96,4789}                                                 & \multicolumn{1}{c|}{63,3333} & 4400,6981 & \multicolumn{1}{c|}{81,1111}                                                  & \multicolumn{1}{c|}{62,2222}                                                 & \multicolumn{1}{c|}{61,1111} & 2776,8963 & \multicolumn{1}{c|}{72,1649}                                                  & \multicolumn{1}{c|}{67,1875}                                                 & \multicolumn{1}{c|}{51,0791} & 3538,5967 \\ \hline
Partición 5-1 & \multicolumn{1}{c|}{98,9437}                                                  & \multicolumn{1}{c|}{96,4912}                                                 & \multicolumn{1}{c|}{50,0000} & 4389,9218 & \multicolumn{1}{c|}{79,4444}                                                  & \multicolumn{1}{c|}{68,8889}                                                 & \multicolumn{1}{c|}{62,2222} & 2802,4383 & \multicolumn{1}{c|}{74,4792}                                                  & \multicolumn{1}{c|}{65,4639}                                                 & \multicolumn{1}{c|}{48,9209} & 3524,8939 \\ \hline
Partición 5-2 & \multicolumn{1}{c|}{98,9474}                                                  & \multicolumn{1}{c|}{97,5352}                                                 & \multicolumn{1}{c|}{43,3333} & 4422,8620 & \multicolumn{1}{c|}{80,0000}                                                  & \multicolumn{1}{c|}{68,3333}                                                 & \multicolumn{1}{c|}{62,2222} & 2812,4940 & \multicolumn{1}{c|}{74,2268}                                                  & \multicolumn{1}{c|}{67,1875}                                                 & \multicolumn{1}{c|}{56,8345} & 3471,6671 \\ \hline
Media         & \multicolumn{1}{c|}{98,7698}                                                  & \multicolumn{1}{c|}{95,8529}                                                 & \multicolumn{1}{c|}{51,0000} & 4426,0345 & \multicolumn{1}{c|}{76,4444}                                                  & \multicolumn{1}{c|}{69,6667}                                                 & \multicolumn{1}{c|}{59,1111} & 2801,2051 & \multicolumn{1}{c|}{74,6633}                                                  & \multicolumn{1}{c|}{65,3904}                                                 & \multicolumn{1}{c|}{54,4604} & 3544,2698 \\ \hline
\end{tabular}}
\end{table}


\begin{table}[H]
\centering
\caption{Resultados BTE}
\label{Resultados BTE}
\resizebox{\textwidth}{!}{\begin{tabular}{|c|cccc|cccc|cccc|}
\hline
              &                                                                               & Wdbc                                                                         &                              &           &                                                                               & \begin{tabular}[c]{@{}c@{}}Movement\\ Libras\end{tabular}                    &                              &           &                                                                               & Arrhythmia                                                                   &                              &           \\ \cline{2-13} 
              & \multicolumn{1}{c|}{\begin{tabular}[c]{@{}c@{}}\%\_clas\\ train\end{tabular}} & \multicolumn{1}{c|}{\begin{tabular}[c]{@{}c@{}}\%\_clas\\ test\end{tabular}} & \multicolumn{1}{c|}{\%\_red} & T (s)     & \multicolumn{1}{c|}{\begin{tabular}[c]{@{}c@{}}\%\_clas\\ train\end{tabular}} & \multicolumn{1}{c|}{\begin{tabular}[c]{@{}c@{}}\%\_clas\\ test\end{tabular}} & \multicolumn{1}{c|}{\%\_red} & T (s)     & \multicolumn{1}{c|}{\begin{tabular}[c]{@{}c@{}}\%\_clas\\ train\end{tabular}} & \multicolumn{1}{c|}{\begin{tabular}[c]{@{}c@{}}\%\_clas\\ test\end{tabular}} & \multicolumn{1}{c|}{\%\_red} & T (s)     \\ \hline
Partición 1-1 & \multicolumn{1}{c|}{99,2958}                                                  & \multicolumn{1}{c|}{94,3860}                                                 & \multicolumn{1}{c|}{53,3333} & 4891,4500 & \multicolumn{1}{c|}{68,8889}                                                  & \multicolumn{1}{c|}{67,2222}                                                 & \multicolumn{1}{c|}{58,8888} & 3159,3639 & \multicolumn{1}{c|}{75,0000}                                                  & \multicolumn{1}{c|}{65,4639}                                                 & \multicolumn{1}{c|}{52,1583} & 4398,2495 \\ \hline
Partición 1-2 & \multicolumn{1}{c|}{97,5439}                                                  & \multicolumn{1}{c|}{96,8310}                                                 & \multicolumn{1}{c|}{20,0000} & 4855,4829 & \multicolumn{1}{c|}{75,5556}                                                  & \multicolumn{1}{c|}{72,2222}                                                 & \multicolumn{1}{c|}{46,6666} & 3183,8460 & \multicolumn{1}{c|}{69,0722}                                                  & \multicolumn{1}{c|}{63,0208}                                                 & \multicolumn{1}{c|}{51,7986} & 4597,1095 \\ \hline
Partición 2-1 & \multicolumn{1}{c|}{97,8873}                                                  & \multicolumn{1}{c|}{96,8421}                                                 & \multicolumn{1}{c|}{30,0000} & 4932,8456 & \multicolumn{1}{c|}{76,6667}                                                  & \multicolumn{1}{c|}{76,6667}                                                 & \multicolumn{1}{c|}{46,6666} & 3251,4145 & \multicolumn{1}{c|}{70,8333}                                                  & \multicolumn{1}{c|}{60,8247}                                                 & \multicolumn{1}{c|}{44,2446} & 4412,4157 \\ \hline
Partición 2-2 & \multicolumn{1}{c|}{99,2982}                                                  & \multicolumn{1}{c|}{93,6620}                                                 & \multicolumn{1}{c|}{40,0000} & 4889,1054 & \multicolumn{1}{c|}{70,0000}                                                  & \multicolumn{1}{c|}{67,2222}                                                 & \multicolumn{1}{c|}{50,0000} & 3174,5812 & \multicolumn{1}{c|}{74,2268}                                                  & \multicolumn{1}{c|}{67,7083}                                                 & \multicolumn{1}{c|}{48,9209} & 4466,9630 \\ \hline
Partición 3-1 & \multicolumn{1}{c|}{98,9437}                                                  & \multicolumn{1}{c|}{96,1404}                                                 & \multicolumn{1}{c|}{33,3333} & 4874,8818 & \multicolumn{1}{c|}{72,7778}                                                  & \multicolumn{1}{c|}{70,5556}                                                 & \multicolumn{1}{c|}{50,0000} & 3163,0693 & \multicolumn{1}{c|}{75,0000}                                                  & \multicolumn{1}{c|}{65,4639}                                                 & \multicolumn{1}{c|}{52,5180} & 4320,1314 \\ \hline
Partición 3-2 & \multicolumn{1}{c|}{98,2456}                                                  & \multicolumn{1}{c|}{96,1268}                                                 & \multicolumn{1}{c|}{43,3333} & 4837,6802 & \multicolumn{1}{c|}{73,3333}                                                  & \multicolumn{1}{c|}{73,3333}                                                 & \multicolumn{1}{c|}{45,5555} & 3183,1151 & \multicolumn{1}{c|}{74,7423}                                                  & \multicolumn{1}{c|}{67,7083}                                                 & \multicolumn{1}{c|}{55,3957} & 4414,9784 \\ \hline
Partición 4-1 & \multicolumn{1}{c|}{97,8873}                                                  & \multicolumn{1}{c|}{96,8421}                                                 & \multicolumn{1}{c|}{40,0000} & 4682,4851 & \multicolumn{1}{c|}{71,6667}                                                  & \multicolumn{1}{c|}{77,7778}                                                 & \multicolumn{1}{c|}{60,0000} & 3161,6695 & \multicolumn{1}{c|}{73,4375}                                                  & \multicolumn{1}{c|}{64,9485}                                                 & \multicolumn{1}{c|}{44,2446} & 4255,8287 \\ \hline
Partición 4-2 & \multicolumn{1}{c|}{98,2456}                                                  & \multicolumn{1}{c|}{97,8873}                                                 & \multicolumn{1}{c|}{56,6667} & 4468,9579 & \multicolumn{1}{c|}{81,1111}                                                  & \multicolumn{1}{c|}{62,7778}                                                 & \multicolumn{1}{c|}{62,2222} & 3134,6418 & \multicolumn{1}{c|}{72,6804}                                                  & \multicolumn{1}{c|}{66,6667}                                                 & \multicolumn{1}{c|}{51,0791} & 4119,7669 \\ \hline
Partición 5-1 & \multicolumn{1}{c|}{98,2394}                                                  & \multicolumn{1}{c|}{94,7368}                                                 & \multicolumn{1}{c|}{53,3333} & 4478,3304 & \multicolumn{1}{c|}{76,6667}                                                  & \multicolumn{1}{c|}{67,7778}                                                 & \multicolumn{1}{c|}{50,0000} & 3182,7247 & \multicolumn{1}{c|}{70,8333}                                                  & \multicolumn{1}{c|}{64,4330}                                                 & \multicolumn{1}{c|}{48,5612} & 4185,7275 \\ \hline
Partición 5-2 & \multicolumn{1}{c|}{98,5965}                                                  & \multicolumn{1}{c|}{97,1831}                                                 & \multicolumn{1}{c|}{43,3333} & 4373,2153 & \multicolumn{1}{c|}{80,0000}                                                  & \multicolumn{1}{c|}{70,0000}                                                 & \multicolumn{1}{c|}{58,8889} & 3164,9642 & \multicolumn{1}{c|}{69,0722}                                                  & \multicolumn{1}{c|}{65,1042}                                                 & \multicolumn{1}{c|}{49,2806} & 4117,4000 \\ \hline
Media         & \multicolumn{1}{c|}{98,4183}                                                  & \multicolumn{1}{c|}{96,0638}                                                 & \multicolumn{1}{c|}{41,3333} & 4728,4435 & \multicolumn{1}{c|}{74,6667}                                                  & \multicolumn{1}{c|}{70,5556}                                                 & \multicolumn{1}{c|}{52,8889} & 3175,9390 & \multicolumn{1}{c|}{72,4898}                                                  & \multicolumn{1}{c|}{65,1342}                                                 & \multicolumn{1}{c|}{49,8201} & 4328,8571 \\ \hline
\end{tabular}}
\end{table}



\subsection{Análisis de los resultados}

\section{Bibliografía}


\end{document}